\documentclass[12pt]{article}
\usepackage[utf8]{inputenc}
\usepackage[romanian]{babel}
\usepackage{hyperref}
\usepackage{amsmath}
\usepackage{amsfonts}
\usepackage{amssymb}
\usepackage{amsthm}
\usepackage{graphicx}
\graphicspath{ {./imagini/} }
\usepackage{pgfplots}
\pgfplotsset{width=10cm,compat=1.9}
\usepackage{pgfplotstable}
\usepackage{booktabs}
\usepackage{array}
\usepackage{colortbl}

\usepackage{listings}
\usepackage{xcolor}

\definecolor{codegreen}{rgb}{0,0.6,0}
\definecolor{codegray}{rgb}{0.5,0.5,0.5}
\definecolor{codepurple}{rgb}{0.58,0,0.82}
\definecolor{backcolour}{rgb}{0.95,0.95,0.92}

\lstdefinestyle{mystyle}{
    backgroundcolor=\color{backcolour},   
    commentstyle=\color{codegreen},
    keywordstyle=\color{magenta},
    numberstyle=\tiny\color{codegray},
    stringstyle=\color{codepurple},
    basicstyle=\ttfamily\footnotesize,
    breakatwhitespace=false,         
    breaklines=true,                 
    captionpos=b,                    
    keepspaces=true,                 
    numbers=left,                    
    numbersep=5pt,                  
    showspaces=false,                
    showstringspaces=false,
    showtabs=false,                  
    tabsize=2
}

\lstset{style=mystyle}

\renewcommand\lstlistingname{Fragment de cod}
\renewcommand\lstlistlistingname{Fragmente de cod}

\title{Comparție teoretică și experimentală a câtorva algoritmi de sortare}
\author{Alexandru Ștefănescu\\
Departamentul de Informatică\\
Facultatea de Matematică şi Informatică, \\
Universitatea de Vest Timişoara, România, \\
Email: \href{mailto:alexandru.stefanescu04@e-uvt.ro}{\texttt{alexandru.stefanescu04@e-uvt.ro}} \date{}
}

\pgfplotstableset{
  every head row/.style={before row=\toprule,after row=\midrule},
  every last row/.style={after row=\bottomrule},
  fixed,precision=2,
}

\begin{document}

\maketitle
\begin{abstract}
Sortarea este o problemă fundamentală în domeniul informaticii, ea stă la baza multor probleme fie că avem nevoie de o listă de numere care să fie în ordine 
crescătoare, fie că ordonăm o listă de persoane, după nume, în ordine alfabetică. Cum volumul informației este în continuă creștere, este foarte
important să putem identifica algoritmul de sortare potrivit pentru nevoile noastre și calitățile și dezavantajele acestuia.

În această lucrare vom analiza 9 algoritmi de sortare, de diferite feluri, și le vom compara performanța pentru
a putea vedea în ce situații se pot folosi fiecare. De asemenea vom vedea conceptele din spatele fiecăruia, cu scopul de a înțelege
și de ce în anumite situații trebuie folosiți anumiți algoritmi.

\end{abstract}

\pagebreak

\tableofcontents

\pagebreak

\section{Introducere}
Sortarea este procesul prin care aranjăm un grup de elemente după un anumit criteriu. De exemplu
ordonarea crescătoare este sortarea elemntelor în funție de valoarea cea mai mare.

De asemenea sortarea stă la baza mai multor operații în domeniul informaticii, precum căutarea.


Mai multe detalii în legătura cu codul sursă se pot găsi pe următorul link:

\url{https://github.com/r0mb-s/Sorting-Algorithms-Benchmark}.


\subsection{Motivaţie}

Sortarea este o operație foarte importantă și, astfel este vital să cunoaștem pe deplin particularitățiile
celor mai importanți algoritmi de sortare, pentru ca mai departe să ne ușurăm munca prin a face alegerea potrivită în funcție de situația care apare.

\section{Descrierea sortării}
\label{sec:formala}

Orice algoritm de sortare poate fi caracterizat prin complexitatea temporală, complexitate spațială, stabilitate, tip de operații,
și adaptabilitate.

\begin{description}
\item[Complexitate temporală: ]
Complexitatea temporală reprezintă timpul în care un anumit algoritm termină de executat ce a avut de făcut. Este reprezentată ca o funcție
care are legătură cu numărul de elemente din datele de intrare. Cum datele de intrare influențează timpul de execuție al algoritmilor
de sortare, funcția care reprezintă complexitatea temporală este notată cu $O$ (ex. $O(n^2)$, unde n este lungimea listei care trebuie sortată) și
$O(n)$ reprezintă o limită superioară a timpului de execuție al agoritmului \cite{cormenIntroduction}.
Astfel un algoritm de sortare pătratic are complexitatea temporală $O(n^2)$, iar unul logaritmic are complexitatea temporală $O(log n)$.
\end{description}

\begin{description}
\item[Complexitate spațială: ]
Complexitatea spațială reprezintă spațiul datelor de intrare și spațiul auxiliar ocupat în memorie pe parcursul execuției
algoritmului Aceasta este de asemenea notată cu $O$ și are ca parametru mărimea datele de intrare. După acest criteriu algoritmii de sortare
pot fi de tip ``pe loc''(in-place, dacă nu folosesc spațiu auxiliar pentru a sorta lista) sau nu.
\end{description}

\begin{description}
\item[Stabilitate: ]
Reprezintă modul în care algoritmul de sortare tratează cazurile în care 2 elemente au aceeași valoare. Un algoritm de sortare este
stabil dacă pastrează ordinea relativă a elementelor de aceeași valoare
\end{description}

\begin{description}
\item[Tip de operații: ]
Tipul de operații reprezintă modul în care un algoritm de sortare funcționeaza. Astfel, avem 2 categorii: algortimii de sortare bazați
pe comparații și cei care nu sunt bazați pe operații. Cei bazați pe comparații compară elementele după valoarea lor, cele care nu sunt
bazate pe comparații folosesc alte metode precum ar fi vectori de frecvență pentru a putea sorta datele de intrare.
\end{description}


\subsection{Preliminarii}
\label{sub:prel}

În această lucrare vom analiza următorii algoritmi de sortare bazați pe comparații (algoritmii au fost preluați din cărțile
\cite{cormenIntroduction}\cite{skiena1998algorithm}\cite{sedgewick1998algorithms}):\newline

{\bf Bubble Sort} - Este o metodă de sortare care parcurge datele de intrare și interschimbă două poziții alăturate dacă cele două
numere nu sunt în ordinea care trebuie. Această parcurgere se repetă până când nu mai este nevoie de nici o interschimbare.

\underline {Proprietăți:}

Complexitate temporală $O(n^2)$

Complexitate spațială $O(n)$ (in-place)

Stabilitate: stabil\newline

{\bf Selection Sort} - Este o metodă de sortare care parcurge datele de intrare și alege cea mai mică valoare din porțiunea
nesortată a listei și o mută în porțiunea sortată.

\underline {Proprietăți:}

Complexitate temporală $O(n^2)$

Complexitate spațială $O(n)$ (in-place)

Stabilitate: instabil (poate fi modificat pentru a fi stabil)\newline

{\bf Insertion Sort} - Această metodă împarte datele de intrare într-o parte sortată și o parte nesortată, iar dupa aceea
extrage câte un element din partea nesortată și îl așează pe poziția sa corespunzătoare în partea sortată.

\underline {Proprietăți:}
    
Complexitate temporală $O(n^2)$

Complexitate spațială $O(n)$ (in-place)

Stabilitate: stabil\newline

{\bf Merge Sort} - Această metodă împarte datele de intrare în două jumătăți, iar apoi sortează fiecare jumătate folosind aceeași
tehnică de divizare până se ajunge la liste de un singur element, după care le reunește recursiv pe toate, într-un final formându-se
lista de la început dar sortată.

\underline {Proprietăți:}

Complexitate temporală $O(n log n)$

Complexitate spațială $O(n)$

Stabilitate: stabil\newline

{\bf Quick Sort} - Această metodă partiționează lista în funcție de un pivot ales, sortează partițiile și le reunește pentru a forma lista sortată.

\underline {Proprietăți:}

Complexitate temporală $O(n log n)$

Complexitate spațială $O(n)$ (in-place)

Stabilitate: instabil (poate fi modificat pentru a fi stabil)\newline

{\bf Heap Sort} - Această metodă metodeă se bazează pe Heapuri binare și de proprietățile acestora \cite{cormenIntroduction} pg. 151

\underline {Proprietăți:}

Complexitate temporală $O(n log n)$

Complexitate spațială $O(n)$ (in-place)

Stabilitate: instabil (poate fi modificat pentru a fi stabil)\newline

{\bf Shell Sort} - Această metodă metodeă sortează perechi de numere îndepărtate (lugimea k) una față de cealaltă
, astfel dupa o sortare lista este k-sortată. După aceea k scade până ajunge la 1, astfel obținându-se o listă sortată.

\underline {Proprietăți:}

Complexitate temporală $O(n log n)$

Complexitate spațială $O(n)$ (in-place)

Stabilitate: instabil (poate fi modificat pentru a fi stabil)\newline

{\bf Tim Sort} - Această metodă împarte lista în segmente pe care le sortează folosind Insertion Sort și după le reunește
folosind merge-ul de la Merge Sort.

\underline {Proprietăți:}

Complexitate temporală $O(n log n)$

Complexitate spațială $O(n)$

Stabilitate: stabil\newline

Și următorii algoritmi care nu sunt bazați pe comparații:

\begin{enumerate}
    \item {\bf Radix Sort} - Este o metodă care folosește Counting Sort sortând fiecare element dupa fiecare cifră (prima oară cifra unităților,
     după cifra zecilor și tot așa).

    \underline {Proprietăți:}
    
    Complexitate temporală $O(n * k)$, unde k este cardinalul alfabetului din care sunt formate numerele

    Complexitate spațială $O(n + k)$

    Stabilitate: stabil
\end{enumerate}

\section{Modelare şi implementare}

Pentru a putea compara algoritmii, și a vedea dincolo de complexitatea temporală descrisă în O, vom
avea nevoie de un generator de numere random pe care le vom adăuga într-o listă. Pentru ca rezultatele să
fie cât mai aproape de realitate vom considera mai multe liste de același număr de elemente.\newline

Pentru cele de mai sus avem următoarea secvență de cod:

\begin{lstlisting}[language=C++]
    for(int l = 1; l <= nrListe; l ++){
        int *lista = new int[nrElementeLista];
        if(lista != nullptr)
            for(int i = 1; i <= nrElementeLista; i ++)
                *(lista + i) = rand() % interval;
    }
\end{lstlisting}
nrListe - reprezintă numărul de liste aleatorii pe care le vom crea\newline
nrElementeLista - reprezintă ce lungime va avea fiecare listă\newline
interval - reprezintă intervalul în care se va încadra fiecare element
din listă.\newline

După ce se crează o listă vom porni un cronometru cu ajutorul librăriei chrono (din C++) și îl vom opri
după terminarea sortării. Acest timp se va adăuga la un total (doar dacă sortarea a avut loc cu succes), iar
după, acel total se va împărți la nrListe pentru a afla timpul mediu de execuție al acelui algoritm.

Pentru a folosi programul, compilați fișierul main.cpp, iar apoi dintr-o linie de comandă rulați executabilul
main cu următorii parametri:

\begin{enumerate}
    \item Primul argument reprezintă algoritmul pe care doriți să îl testați și se vor folosi cuvintele cheie:
    
    ``bubble'' (Bubble Sort), ``insert'' (Insertion Sort), ``select'' (Selection Sort), ``quick'' (QuickSort),
    ``merge'' (Merge Sort), ``heap'' (Heap Sort), ``shell'' (Shell Sort), ``tim'' (Tim Sort), ``count'' (Counting Sort),
    ``radix'' (Radix Sort).

    \item Al doilea argument reprezintă lungimea listei. Aceasta se va introduce ca un numă natural $n$, unde lungimea listei
    va fi $10^n$.

    \item Al treilea argument reprezintă numărul de liste. Aceasta se va introduce ca un numă natural $n$, unde lungimea listei
    va fi $10^n$.

    \item Al patrulea argument reprezintă intervalul elementelor din listă. Dacă valoarea trecută este 0,
    atunci intervalul va fi constanta RAND\_MAX.

    \item Al cincilea argument reprezintă starea datelor de intrare, 0 - elemente random, 1 - elemente sortate crescător
    , -1 - elemente sortate descrescător. 

\end{enumerate}

Un exemplu de sir de argumente ar fi:\newline

{\bf main radix 6 2 100 -1} \newline

Care va sorta $10^2$ liste ordonate descrescător, fiecare de câte $10^6$ elemente și fiecare elemnt aflându-se în intervalul $[0, 100]$


\section{Studiu de caz / experiment}

Pentru fiecare algoritm s-au efectuat teste cu liste de 1.000 până la 1.000.000 de elemente pentru algoritmii
pătratici și de la 1.000 până la 1.000.000.000 pt cei logaritmici și cei care nu sunt bazați pe comparații.
Tabelele cu rezultatele se pot găsi în anexe.

Testele au fost făcute pe un procesor AMD® Ryzen 7 4800H.
\pagebreak

\begin{figure}[!h]
    \centering
    \begin{tikzpicture}
\begin{loglogaxis}[
    width=15cm,
    xlabel={Date de intrare},
    ylabel={Timp (s)},
    xmin=9, xmax=1500000,
    ymin=0, ymax=4000,
    xtick={10, 100, 1000, 10000, 100000, 1000000},
    ytick={0.000001, 0.00001, 0.0001, 0.001, 0.01, 0.1, 1, 10, 100, 1000, 10000},
    legend pos=north west,
    ymajorgrids=true,
    grid style=dashed,
]
      
\addplot table [x=ele,y=bubble] {grafic1.txt};
\legend{Bubble Sort}
\addplot table [x=ele,y=insertion] {grafic1.txt};
\addlegendentry{Insertion Sort}
\addplot table [x=ele,y=selection] {grafic1.txt};
\addlegendentry{Selection Sort}
\addplot table [x=ele,y=tim] {grafic1.txt};
\addlegendentry{Tim Sort}

    
\end{loglogaxis}
\end{tikzpicture}
\end{figure}

Rezultatele de mai sus au fost obținute analizând comportamentul celor 4 algoritmi
pe liste nesortate cu elemente aleatorii din intervalul $0$ - $2^{32}$. Am mai adăugat pe
lângă Bubble Sort, Insertion Sort și Selction Sort un algoritm logaritmic pentru a ne putea
crea o imagine de ansamblu asupra diferenței dintre cele două tipuri.

De asemenea, putem vedea că pentru liste de lungime trivială timpii de execuție sunt foarte
asemănători indiferent de metoda de sortare. În plus putem vedea că deși cei 3 algoritmi au complexitatea
temporală $O(n^2)$, Insertion Sort este cel mai rapid dintre ei, având la liste de până la 100
de elemente același timp de execuție ca și Tim Sort și menținând un avantaj semnificativ și constant față
de Bubble Sort și Selection Sort.

\pagebreak

\begin{figure}[!h]
    \centering
    \begin{tikzpicture}
\begin{loglogaxis}[
    width=15cm,
    xlabel={Date de intrare},
    ylabel={Timp (s)},
    xmin=9, xmax=1500000,
    ymin=0, ymax=4000,
    xtick={10, 100, 1000, 10000, 100000, 1000000},
    ytick={0.000001, 0.00001, 0.0001, 0.001, 0.01, 0.1, 1, 10, 100, 1000, 10000},
    legend pos=north west,
    ymajorgrids=true,
    grid style=dashed,
]
      
\addplot table [x=ele,y=bubble(a)] {grafic1.txt};
\legend{Bubble Sort}
\addplot table [x=ele,y=insert(a)] {grafic1.txt};
\addlegendentry{Insertion Sort}
\addplot table [x=ele,y=select(a)] {grafic1.txt};
\addlegendentry{Selection Sort}
\addplot table [x=ele,y=tim(a)] {grafic1.txt};
\addlegendentry{Tim Sort}

    
\end{loglogaxis}
\end{tikzpicture}
\end{figure}

Deși pare a fi un grafic total diferit, singura diferență sunt datele de intrare. Putem vedea o schimbare majoră
atunci când lista este deja sortată crescător, în special Insertion Sort reușește să se execută mai repede chiar decât Tim Sort,
acesta atingând un timp aproape liniar, dar nu numai, toți algoritmii văd o îmbunătățire a timpului de execuție.

Insertion Sort prezintă acest comportament deoarece fiecare element nou pe care îl compară cu anteriorul (cel mai mare din partea sortată)
este mai mare și astfel nu se efectuează nicio interschimbare. Un astfel de comportament l-ar putea prezenta și Bubble Sort
dacă am aplica o mică optimizare, aceea fiind ca la fiecare parcurgere să se verifice dacă nu cumva lista este deja sortată.



\pagebreak

\begin{figure}[!h]
    \centering
    \begin{tikzpicture}
\begin{loglogaxis}[
    width=15cm,
    xlabel={Date de intrare},
    ylabel={Timp (s)},
    xmin=9, xmax=1500000,
    ymin=0, ymax=4000,
    xtick={10, 100, 1000, 10000, 100000, 1000000},
    ytick={0.000001, 0.00001, 0.0001, 0.001, 0.01, 0.1, 1, 10, 100, 1000, 10000},
    legend pos=north west,
    ymajorgrids=true,
    grid style=dashed,
]
      
\addplot table [x=ele,y=bubble(d)] {grafic1.txt};
\legend{Bubble Sort}
\addplot table [x=ele,y=insert(d)] {grafic1.txt};
\addlegendentry{Insertion Sort}
\addplot table [x=ele,y=select(d)] {grafic1.txt};
\addlegendentry{Selection Sort}
\addplot table [x=ele,y=tim(d)] {grafic1.txt};
\addlegendentry{Tim Sort}

    
\end{loglogaxis}
\end{tikzpicture}
\end{figure}

Spre deosebire de schimbarea anterioară, algoritmi râmân cam pe aceleași locuri având ca date de intrare
o listă sortată descrescător. În această situație Insert Sort nu mai intră în situația cea mai favorabilă, ci chiar în cea mai
nefavorabilă, dublându-și aproape timpul de execuție în aproape toate lungimile de listă.

\pagebreak

\begin{figure}[!h]
    \centering
    \begin{tikzpicture}
\begin{loglogaxis}[
    width=15cm,
    xlabel={Date de intrare},
    ylabel={Timp (s)},
    xmin=100, xmax=101000000,
    ymin=0.000001, ymax=100,
    xtick={10, 100, 1000, 10000, 100000, 1000000, 10000000, 100000000, 1000000000},
    ytick={0.000001, 0.00001, 0.0001, 0.001, 0.01, 0.1, 1, 10, 100, 1000, 10000},
    legend pos=north west,
    ymajorgrids=true,
    grid style=dashed,
]
      
\addplot table [x=ele,y=quick] {grafic2.txt};
\legend{Quick Sort}
\addplot table [x=ele,y=merge] {grafic2.txt};
\addlegendentry{Merge Sort}
\addplot table [x=ele,y=heap] {grafic2.txt};
\addlegendentry{Heap Sort}
\addplot table [x=ele,y=shell] {grafic2.txt};
\addlegendentry{Shell Sort}
\addplot table [x=ele,y=tim] {grafic2.txt};
\addlegendentry{Tim Sort}
\addplot table [x=ele,y=radix(2)] {grafic2.txt};
\addlegendentry{Radix Sort(100)}
\addplot table [x=ele,y=radix(6)] {grafic2.txt};
\addlegendentry{Radix Sort(1 000 000)}
\addplot table [x=ele,y=radix] {grafic2.txt};
\addlegendentry{Radix Sort}

    
\end{loglogaxis}
\end{tikzpicture}
\end{figure}

După cum putem vedea din grafic Radix Sort pe elemente din intervalul [0, 100] este cel mai rapid, Tim Sort fiind următorul.
Pe cealaltă parte avem Radix Sort pe elemente din intervalul [0, $2^{32}$] care pare să fie cel mai încet pe setul de date pe care l-am testat până
ajungem la liste de $10^8$ elemente, unde vedem ca Heap Sort durează cel mai mult.

Deși Radix Sort pare să fie cel mai rapid, o situație în care să avem de sortat elemente doar din intervalul [0, 100] este
foarte rară și astfel l-am introdus doar pentru a putea vedea avantajul acestui algoritm atunci când intervalul elementelor
se schimbă

\pagebreak


\section{Comparaţia cu literatura}

Lucruri similare au fost făcute în~\cite{al2013review}. Deși la prima vedere rezultatele par foarte diferite,
când luam în vedere diferența de hardware și ne uitam mai mult la diferența dintre algoritmi putem observa aproximativ
aceleași proporții numai că la liste de numere mult mai scurte.

Alt articol care prezintă același lucru este \cite{hammad2015comparative}, având pe lângă partea experimentală și o parte
teoretică mult mai bogată în conținut, conținând metode de creare a unor algoritmi eficienți,
metode de abordare a problemei sortării și descrierea unor algoritmi cunoscuți.

\section{Concluzii şi direcţii viitoare}

Sortarea reprezintă un capitol foarte important al informaticii, dar și foarte complex. Există
o mulțime de algoritmi pe care nu am reușit să îi descriu în această lucrare, care au avantajele și dezavantajele
lor și care pot avea comportamente surprinzătoare în anumite situații, cum a fost și Insert Sort
la liste deja sortate.

Astfel, pe viitor am de gând să îmi extind orizontul printr-o lucrare mult mai complexă, nu neapărat prin numărul de algoritmi
pe care să îi analizez, dar mai degrabă pe diversitatea lor. De exemplu, abordarea algoritmilor de sortare
paraleli, dar și algoritmi care au nevoie de hardware specializat pentru a atinge potențialul maxim precum Bead Sort.

\pagebreak
\bibliographystyle{apalike}
\bibliography{bibliografie}

\pagebreak

\section{Anexe}

\begin{table}[h!]
    \centering
    \begin{tabular}{ ||r| c| c| c|| }
\hline\hline
  Elemente & Bubble sort & Insertion Sort & Selection Sort  \\
  \hline
10  & 0.0000018 & 0.0000017 & 0.0000016 \\
100  & 0.0000383 & 0.0000241 & 0.0000107 \\
1000  & 0.001582 & 0.001008 & 0.000439\\
10000  & 0.17490 & 0.095847 & 0.04624 \\
100000  & 25.293 & 9.5150 & 5.0201 \\
1000000  & 2596.251  & 952.793 & 580.8731 \\
\hline\hline
\end{tabular}
    \caption{Rezultate experimentale obţinute (în secunde).}
\end{table}

\begin{table}[h!]
    \centering
    \begin{tabular}{ ||r| c| c| c|| }
\hline\hline
  Elemente & Bubble sort & Insertion Sort & Selection Sort  \\
  \hline
10  & 0.0000012 & 0.0000011 & 0.0000011 \\
100  & 0.0000123 & 0.0000114 & 0.0000013 \\
1000  & 0.001072 & 0.000962 & 0.000003 \\
10000  & 0.10636 & 0.09572 & 0.000026 \\
100000  & 10.6198 & 9.5336 & 0.00025 \\
1000000  & 1064.217 & 956.002 & 0.0025 \\
\hline\hline
\end{tabular}
    \caption{Rezultate experimentale obţinute - listă sortată crescător (în secunde).}
\end{table}

\begin{table}[h!]
    \centering
    \begin{tabular}{ ||r| c| c| c|| }
\hline\hline
  Elemente & Bubble sort & Insertion Sort & Selection Sort  \\
  \hline
10  & 0.0000013 & 0.0000012 & 0.0000012 \\
100  & 0.0000215 & 0.0000103 & 0.0000125 \\
1000  & 0.001939 & 0.000882 & 0.001017 \\
10000  & 0.19335 & 0.09752 & 0.10006 \\
100000  & 19.403 & 9.8028 & 10.1474 \\
1000000  & 1938.268 & 1192.13 & 1036.159 \\
\hline\hline
\end{tabular}
    \caption{Rezultate experimentale obţinute - listă sortată descrescător (în secunde).}
\end{table}

\begin{table}[h!]
    \centering
    \begin{tabular}{ ||r| c| c| c| c| c|| }
\hline\hline
  Elemente & Quick sort & Merge Sort & Heap Sort & Shell Sort & Tim Sort \\
  \hline
10  & 0.0000014 & 0.0000016 & 0.0000014 & 0.0000013 & 0.0000015 \\
100  & 0.0000060 & 0.0000090 & 0.0000080 & 0.0000060 & 0.0000048 \\
1000  & 0.000073 & 0.000109 & 0.000110 & 0.000085 & 0.000065 \\
10000  & 0.000976 & 0.0013412 & 0.0014331 & 0.0012291 & 0.0009048 \\
100000  & 0.01221 & 0.015988 & 0.018105 & 0.017746 & 0.011647 \\
1000000  & 0.14685 & 0.18627 & 0.25131 & 0.25654 & 0.14187 \\
10000000  & 1.7152 & 2.13423 & 3.96204 & 3.88753 & 1.70757 \\
100000000  & 19.574 & 24.153 & 58.732 & 20.487 & 20.097 \\
\hline\hline
\end{tabular}
    \caption{Rezultate experimentale obţinute (în secunde).}
\end{table}

\end{document}